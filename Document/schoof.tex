\documentclass[letterpaper, 12pt]{article}
\usepackage[french]{babel}
\usepackage[utf8x]{inputenc}
\usepackage{amsmath}
\usepackage{graphicx}
\usepackage[colorinlistoftodos]{todonotes}
\usepackage[margin=1in,letterpaper]{geometry}

\usepackage{lmodern}
\usepackage{amssymb}
\usepackage{mathrsfs}
\usepackage[T1]{fontenc}
\usepackage{pifont}
\usepackage{enumitem}
\usepackage{color}
\usepackage{algorithm2e}
\usepackage{stmaryrd}
\usepackage{graphicx}
\usepackage{verbatim}
\usepackage{fancyhdr}
\usepackage{tabularx}
\usepackage{listings}
\usepackage{fancyhdr}
\usepackage{listings}
\usepackage{hyperref}


\newcommand{\Z}{\mathbb{Z}}
\newcommand{\F}{\mathbb{F}}
\newcommand{\N}{\mathbb{N}}
\newcommand{\R}{\mathbb{R}}

\begin{document}

\begin{titlepage}

\newcommand{\HRule}{\rule{\linewidth}{0.5mm}} % Defines a new command for the horizontal lines, change thickness here

\center 
 
%----------------------------------------------------------------------------------------
%	HEADING SECTIONS
%----------------------------------------------------------------------------------------

\textsc{\LARGE Université de Versailles}\\[1.5cm] % Name of your university/college
\textsc{\Large Projet d'étude }\\[0.5cm] % Major heading such as course name
\textsc{\large Master d'algèbre appliquée à la cryptographie}\\[0.5cm] % Minor heading such as course title

%----------------------------------------------------------------------------------------
%	TITLE SECTION
%----------------------------------------------------------------------------------------

\HRule \\[0.4cm]
{ \huge \bfseries Algorithme de comptage de points de Schoof}\\[0.1cm] % Title of your document
\HRule \\[1.5cm]
 
%----------------------------------------------------------------------------------------
%	AUTHOR SECTION
%----------------------------------------------------------------------------------------

\begin{minipage}{0.5\textwidth}
\begin{flushleft} 
\emph{Réalisé par:}\\
Abdourahman \textsc{Saleh Ibrahim}\\
Abdessamad \textsc{Fazzat} 

\end{flushleft}
\end{minipage}
~
\begin{minipage}{0.4\textwidth}
\begin{flushright} 
\emph{Supervisé par:} \\
Prof. Luca \textsc{De Feo} 
\end{flushright}
\end{minipage}\\[2cm]


%----------------------------------------------------------------------------------------
%	DATE SECTION
%----------------------------------------------------------------------------------------
\center {\large \today}\\[2cm]
%----------------------------------------------------------------------------------------
%	LOGO SECTION
%----------------------------------------------------------------------------------------

\includegraphics[scale=0.15]{versailles.jpg} 
%----------------------------------------------------------------------------------------


\vfill %

\end{titlepage}
\tableofcontents
\newpage

\begin{abstract}
Résumé du projet
\end{abstract}

\section{Introduction}
Introduction aux courbes elliptiques et application à la cryptographie

\section{Courbes elliptiques}
    \subsection{Courbes elliptiques}
    
    \subsection{Loi de groupe}
    
    \subsection{Endomorphisme de Frobenius}
    
    \subsection{Nombre de points}
\section{Arithmétique dans un corps fini}
    \subsection{Algorithme d'Euclide}
    \subsection{Algorithme d'Euclide étendu}
    \subsection{Inversion modulaire}
    \subsection{Exponentiation modulaire}
    \subsection{Racine carrée modulaire}
    \subsection{Théorème Chinois des restes}

\section{Cryptographie sur les courbes elliptiques}
    \subsection{Problème du logarithme discret}
    
\section{Algorithme de Schoof}
    \subsection{Polynômes de division}
    \subsection{Points de torsion}
    \subsection{Une borne sur le nombre de points: Théorème de Hasse}
    \subsection{Algorithme de comtage de points de Schoof}
    \subsection{Algorithme de Schoof-Elkies-Atkin}
    
\section{Implémentation de l'algorithme de Schoof}
    \subsection{Complexité}
    
\newpage
\newpage
\begin{thebibliography}{99}
\item R. Schoof. Elliptic curves over finite fields and the computation of square roots mod p. Mathematics of computation 44.170 (1985):483-494.

\item R. Crandall, C. Pomerance. Prime numbers: a computational perspective. Vol. 182. Springer Science & Business Media, 2006. Côte 512.7 CRA. §7.5.2.

\item I. F. Blake, G. Seroussi, N. Smart. Elliptic curves in cryptography. Vol. 265. Cambridge university press, 1999. Côte 005.82 BLA. Chapitre VII.

\item P. Guillot: Introduction aux courbes elliptiques pour la cryptographie.
\end{thebibliography}
\end{document}

\begin{abstract}
Your abstract.
\end{abstract}

\section{Introduction}


\end{document}
